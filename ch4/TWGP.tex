\documentclass[../main.tex]{subfiles}

\begin{document}
	
	\section{Maintaining Treewidth in Graph Products}
	\subsection{Introduction}
	With the importance of treewidth and the many applications of product structure theory, a natural question is the following: Can the treewidth be maintained in some meaningful way through taking the product? Solving this problem in any sense would have deep impact on structural graph theory by giving a strong tool to study the treewidth of complex graph classes by using product structure theorems to simplify the problem and give a new approach to improving the treewidth bounds of complicated graph classes.
	\subsection{Problem 4}
	  Is it true that for every planar graph $G$, there exists a bounded treewidth graph $H$ and a path $P$ such that $G \subsetsim H\boxtimes P$ and $\tw(H\boxtimes P) \in O(\tw(G))$? 
	
	\subsection{Related Work}
	In our paper regarding Problem 1 \cite[Lemma~3, Equation~(2)]{DMWW24}, we show that $$\Omega(\min\{|V(H)|,|V(P)|\})\leq  \tw( H\boxtimes P) \leq O(\min\{|V(H)|,|V(P)|\}).$$ 
	
	Thus we can instead ask whether for every planar graph $G$, there exists a bounded treewidth graph $H$ and a path $P$ such that $G\subsetsim H\boxtimes P$ and $\min\{|V(H)|,|V(P)|\} \leq O(\tw(G))$. 
	
	Little research has been done studying product structures where the full product has bounded treewidth. In fact, it is even open whether $\min\{V(H)|,|V(P)|\} \leq O(f(\tw(G)))$ for some function $f$, or if $\min\{V(H)|,|V(P)|\} \leq O(\sqrt{|V(G)|})$. The second bound is weaker due to the planar separator theorem\cite{LT79}, which states that $\tw(G) \le O(\sqrt{|V(G)|}\})$ for every planar graph $G$.
	
	\bibliographystyle{plain}
	\bibliography{largebib-twgp}
\end{document}