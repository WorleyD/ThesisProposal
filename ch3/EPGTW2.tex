\documentclass[../main.tex]{subfiles}

\begin{document}
	
	\section{Product Structure of Bounded Degree Planar Graphs}
	\subsection{Introduction}
	The usage of product structure theory to study planar graphs has been a very active area of research since Dujmovic et al first showed that planar graphs have bounded queue number using product structure theory \cite{DJMMUW20}. Since then, this result has gone on to lead to improvements in graph colouring\cite{DEJWW20}, adjacency labelling\cite{DEJGMM21,EJM23}, and more. 
	These results have pushed additional interest into the research of product structure theorems for other graph classes and for more specialized ones. We focus on improving the product structure theorems for bounded-degree planar graphs.
	
	\subsection{Problem 3}
	Given a planar graph $G$ with maximum degree $\Delta$, is it true that $G$ is contained in the product $$H\boxtimes P\boxtimes K_c$$ for a graph $H$ of treewidth 2, a path $P$, and the complete graph $K_c$ where $c$ is bounded by some function of $\Delta$?
	
	\subsection{Related Work}
	This problem looks to tighten the bound on $\tw(H)$ to close the bound on the product structure of bounded-degree planar graphs. The problem was initially shown to be true for $\tw(H)=3$ by Dujmovic et al \cite{DJMMUW20}.It was then shown that the case for $\tw(H)=1$ is false\cite{DJMMW24}. 
	In particular, that there exists an infinite family of graphs $\mathbb{G}$ of maximum degree 5 such that for each $g\in\mathbb{G}$, for every graph $H$ of treewidth $t$ and maximum degree $\Delta$, every path $P$, and every integer $c$, if $G\subseteq H \boxtimes P\boxtimes K_c$, then $t\Delta c\ge 2^{\Omega(\sqrt{\log\log n})}$.
	
	\bibliographystyle{plain}
	
	\bibliography{largebib-epgtw2}
\end{document}